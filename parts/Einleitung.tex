\begin{frame}
    \Huge Einleitung
\end{frame}
\begin{frame}
    \frametitle{Einleitung}
    \framesubtitle{Was ist ein Graph}
    \begin{KITexampleblock}{Graph}
\begin{itemize}
    \item Ein Graph $G$ ist ein geordnetes Paar $G = (V, E)$
    \item $V$ Menge von Knoten/Vertices
    \item $E$ Menge von Kanten/Edges
    \begin{itemize}
        \item gerichteter Graph (ohne Mehrfachkanten) $E \subseteq V \times V$
        \item ungerichteter Graph (ohne Mehrfachkanten) $E \subseteq V \times V \hspace{0.2cm} \land \hspace{0.2cm} (u,w) \in E \iff (w,u) \in E$
    \end{itemize}
\end{itemize}
    \end{KITexampleblock}
\end{frame}
% \begin{frame}
%     \frametitle{Einleitung}
%     \framesubtitle{Grundbegriffe der Graphentheorie}
%     \begin{center}
%     einige bereits bekannte Begriffe \\
%     \begin{tabular}{c | c | c}
%     \hline
%     Un/Gerichtet & Un/Gewichtet & Schlinge \\ \hline
%     (Ein/Ausgangs) Grad & Weg/Pfad & Zyklus/Kreis \\ \hline
%     Knoten ereichbar/isoliert & Teil/Untergraph & Baum \\ \hline
%     \end{tabular}
%     \end{center}
%     \begin{itemize}
%     \item einfacher Graph := ungerichteter Graph, ohne Mehrfachkanten und ohne Schleifen
%     \item DAG (directed acyclic graph) := ein gerichteter Graph ohne Zyklus
%     \item vollständiger Graph := maximale Kantenanzahl \( |E| = \frac{|V| (|V| - 1)}{2}\)
%     \item Dichte \( dn(G) = \frac{2|E|}{|V| (|V| - 1)}\)
%     \begin{itemize}
%         \item Verhältnis der Kantenanzahl zur Kantenanzahl eines vollständigen Graphen auf gleichvielen Knoten
%     \end{itemize}
%     \end{itemize}
% \end{frame}

% \begin{frame}
%     \frametitle{Einleitung}
%     \framesubtitle{Grundbegriffe der Graphentheorie}
%     \begin{itemize}
%         \item Teilgraph \( T_V \subseteq V\) und \( T_E \subseteq E \land \forall (x, y) \in T_E : x, y \in T_V\)
%         \item Untegraph \( U_V \subseteq V\) und \( U_E = \{ (x, y) \in E \mid x, y \in U_V\}\)
%         \item Schlinge $\{(x,x) \in E \mid x \in V\}$
%     \end{itemize}
% \end{frame}

% \begin{frame}
%     \frametitle{Einleitung}
%     \framesubtitle{Grundbegriffe der Graphentheorie}
%     \begin{itemize}
%         \item Weg \( w = (v_1, v_2, … , v_n) \mbox{ und } \forall i \in \{1, … , n-1\} : (v_i, v_{i+1}) \in E \)
%         \item Pfad $p :=$ Weg mit paarweise verschiedenen Knoten
%         \item Kreis \( := \) Pfad, der im selben Knoten beginnt und endet
%         \item Zyklus \( :=\) Weg, der im selben Knoten beginnt und endet
%         \item Eulerkreis := Zyklus der über alle Kanten eines Graphen läuft
%     \end{itemize}
% \end{frame}

% \begin{frame}
%     \frametitle{Einleitung}
%     \framesubtitle{Grundbegriffe der Graphentheorie}
%         \begin{KITexampleblock}{Baum}
%     \begin{itemize}
        % \item maximal kreisfrei $\iff$ keine Kante kann zur Kantenmenge hinzugefügt werden, ohne einen Kreis zu erzeugen
        % \item minimal zusammenhängend $\iff$ keine Kannte kann entfernt werden, ohne die Zusammenhangs-Eigenschaft zu verletzen
%     \end{itemize}
    % \begin{KITexampleblock}{Baum}
% \end{frame}

\begin{frame}
    \frametitle{Einleitung}
    \framesubtitle{Beispiel}
    \begin{KITexampleblock}{Ungerichteter Beispielgraph}
    \begin{center}
    \includegraphics[scale=1.4]<1-1>{ipe.pdf}

    % \begin{tabular}{c | c | c | c}
    %  & Adj.matrix & Adj.liste & Edgelist\\ \hline
    % Code & int[][]  & vec<vec<pair<int,int>{}>{}> & \\ \hline
    % Platz & \( \mathcal{O}(|V|^2)\) &\( \mathcal{O}(|V|+|E|)\) &\( \mathcal{O}(|E|)\) \\
    % \end{tabular}
    \end{center}
    \end{KITexampleblock}
\end{frame}

\begin{frame}
    \frametitle{Einleitung}
    \framesubtitle{Implementierungen}
    \begin{KITexampleblock}{Adjazenzmatrix}
    int[][]
    \begin{itemize}
        \item Speicherverbrauch \( \mathcal{O}(|V|^2)\)
        \item über Nachbarn iterieren \( \mathcal{O}(|V|)\)
        \item ICPC fast nur bei Floyd-Warschall
    \end{itemize}
    \end{KITexampleblock}
    \vspace{0.001em}
    \begin{KITinfoblock}{Visualisierung}
        $\bordermatrix{
  & 0  & 1  & 2  & 3  & 4  & 5  \cr
0 & 0  & 23 & 2  & 0  & 0  & 0  \cr
1 & 23 & 0  & 2  & 0  & 0  & 0  \cr
2 & 2  & 2  & 0  & 5  & 0  & 0  \cr
3 & 0  & 0  & 5  & 0  & 3  & 15 \cr
4 & 0  & 0  & 0  & 3  & 0  & 0  \cr
5 & 0  & 0  & 0  & 15 & 0  & 0  \cr
} $ %\iff $ 
    % \includegraphics[scale = 0.4]<1-1>{./images/Einleitung/graph.png}
    \end{KITinfoblock}
\end{frame}

\begin{frame}
    \frametitle{Einleitung}
    \framesubtitle{Implementierungen}
    \begin{KITexampleblock}{Adjazenzlist}
    vector<vector<pair<int,int>{}>{}>
    \begin{itemize}
        \item Speicherverbrauch \( \mathcal{O}(|V|+|E|)\)
        \item kompakter und effizienter, da \( |E| \ll \frac{|V| (|V| - 1)}{2} \in \mathcal{O}(|V|^2)\)
        \item über alle k Nachbarn iterieren \( \mathcal{O}(k)\)
        \item ICPC Standardwahl
    \end{itemize}
    \end{KITexampleblock}
    \vspace{0.001em}
    \begin{KITinfoblock}{Visualisierung}
    \begin{tabular}{l c c c}
        \textbf{0 :  } & (23,1) & (2,2) \\
        \textbf{1 :  } & (23,0) & (2,2) & (5,3) \\
        \textbf{2 :  } & (2,0) & (2,1) \\
        \textbf{3 :  } & (5,1) & (3,4) & (15,3) \\
        \textbf{4 :  } & (3,3) \\
        \textbf{5 :  } & (15,3) \\
    \end{tabular}
    \end{KITinfoblock}
\end{frame}

\begin{frame}
    \frametitle{Einleitung}
    \framesubtitle{Implementierungen}
    \begin{KITexampleblock}{Kantenlist}
    vector<tuple<int,int,int>{}>
    \begin{itemize}
        \item Speicherverbrauch \( \mathcal{O}(|E|)\)
        \item sortiert (nach Gewicht) wichtige Representation
        \item ICPC bei Greedyalgorithmen (z.B. Kruskal)
    \end{itemize}
    \end{KITexampleblock}
    \vspace{0.001em}
    \begin{KITinfoblock}{Visualisierung}
    \begin{tabular}{l c c c}
        \textbf{0 :  } & 23 & 0 & 1 \\
        \textbf{1 :  } & 2 & 0 & 2 \\
        \textbf{2 :  } & 2 & 2 & 1 \\
        \textbf{3 :  } & 5 & 1 & 3 \\
        \textbf{4 :  } & 3 & 3 & 4 \\
        \textbf{5 :  } & 15 & 3 & 5 \\
    \end{tabular}
    \end{KITinfoblock}
\end{frame}
