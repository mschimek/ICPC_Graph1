\begin{frame}
	\frametitle{SCCs finden mittels DFS}
	\framesubtitle{Beispielaufgabe}
	\begin{KITexampleblock}{UVa 11383 Come and Go}
		In einer Stadt gibt es $N$  Kreuzungen, die durch Straßen verbunden sind.
		Da man in der Stadt von einem Punkt (Kreuzung) zu jedem anderen kommen möchte, sollte es eine Verbindung zwischen zwei beliebigen Kreuzungen geben. \\
		Für eine gegebene Stadt mit $N$ Kreuzungen und $M$ Straßen soll entschieden werden, ob dies möglich ist.
			
	\end{KITexampleblock}
	\pause 
	\bigskip
	
	\begin{KITinfoblock}{Definition Strongly Connected Components SCC}
		In einem gerichteten Graph $G = (V,E)$, wird $V' \subseteq V$ starke Zusammenhangskomponente (SCC) genannt, wenn zwischen je zwei Knoten in $V'$ ein Pfad existiert.
	\end{KITinfoblock}
	
	
\end{frame}

\begin{frame}
	\frametitle{SCCs finden mittels DFS}
	\begin{itemize}
		\item Zum Lösen der Aufgabe untersuchen, ob das Straßennetz der Stadt aus einer oder mehreren SCCs besteht.
		\item $\Rightarrow$ Benötigen effizienten Algorithmus zum Finden von SCCs 
	\end{itemize}	
	\pause
	\heading{Algorithmus von Tarjan für SCCs}
	\begin{itemize}
		\item Wurde von Robert Tarjan gefunden
		\item Basiert auf dem Konzept der DFS
		\item Laufzeit: $\mathcal{O}(|V| + |E|)$
	\end{itemize}
\end{frame}
\begin{frame}
	
	\frametitle{SCCs finden mittels DFS}
	\includegraphics[scale = 0.5]<1-1>{SCC.pdf}
	\includegraphics[scale = 0.5]<2-2>{SCC_1.pdf}
	\includegraphics[scale = 0.5]<3-3>{SCC_2.pdf}
	\includegraphics[scale = 0.5]<4-4>{SCC_3.pdf}
	\includegraphics[scale = 0.5]<5-5>{SCC_5.pdf}
	\includegraphics[scale = 0.5]<6-6>{SCC_6.pdf}
	\includegraphics[scale = 0.5]<7-7>{SCC_7_2.pdf}
	\includegraphics[scale = 0.5]<8-8>{SCC_7.pdf}
	\includegraphics[scale = 0.5]<9-9>{SCC_8.pdf}
	\includegraphics[scale = 0.5]<10-10>{SCC_9.pdf}
	\includegraphics[scale = 0.5]<11-11>{SCC_10.pdf}
	\includegraphics[scale = 0.5]<12-12>{SCC_11.pdf}
	\includegraphics[scale = 0.5]<13-13>{SCC_12.pdf}
	\includegraphics[scale = 0.5]<14-14>{SCC_13.pdf}
\end{frame}

\begin{frame}
	\frametitle{SCCs finden mittels DFS}
	\framesubtitle{Sourcecode}
	\lstset{basicstyle=\scriptsize, commentstyle=\color{mygreen},frame=single}
	\lstinputlisting[language=C++,firstline=17, lastline=37]{sourcecode/SCC_all.cpp}
\end{frame}

\begin{frame}
	\begin{itemize} 
		\item \textbf{findSCC(int u)} findet alle SCCs, die von Konten \textbf{u} aus erreichbar sind.
			\item Für vollständige Liste an SCCs \textbf{findSCC(int u)} für alle Knoten eines Graphen laufen lassen.
	\end{itemize}
\end{frame}