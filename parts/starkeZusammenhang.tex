\begin{frame}
	\Huge Starke Zusammenhangskomponenten
\end{frame}

\begin{frame}
	\frametitle{SCCs finden mittels DFS}
	\framesubtitle{Beispielaufgabe}
	\begin{KITexampleblock}{UVa 11383 Come and Go}
		In einer Stadt gibt es $N$  Kreuzungen, die durch Straßen verbunden sind.
		Da man in der Stadt von einem Punkt (Kreuzung) zu jedem anderen kommen möchte, sollte es eine Verbindung zwischen zwei beliebigen Kreuzungen geben. \\
		Für eine gegebene Stadt mit $N$ Kreuzungen und $M$ Straßen soll entschieden werden, ob dies möglich ist.
			
	\end{KITexampleblock}
	\pause 
	\bigskip
	
	\begin{KITinfoblock}{Definition Strongly Connected Components SCC}
		In einem gerichteten Graph $G = (V,E)$, wird $V' \subseteq V$ starke Zusammenhangskomponente (SCC) genannt, wenn zwischen je zwei Knoten in $V'$ ein Pfad existiert.
	\end{KITinfoblock}
	
	
\end{frame}

\begin{frame}
	\frametitle{SCCs finden mittels DFS}
	\begin{itemize}
		\item Zum Lösen der Aufgabe untersuchen, ob das Straßennetz der Stadt aus einer oder mehreren SCCs besteht.
		\item $\Rightarrow$ Benötigen effizienten Algorithmus zum Finden von SCCs 
	\end{itemize}	
	\pause
	\heading{Algorithmus von Tarjan für SCCs}
	\begin{itemize}
		\item Wurde von Robert Tarjan gefunden
		\item Basiert auf dem Konzept der DFS
		\item Laufzeit: $\mathcal{O}(|V| + |E|)$
	\end{itemize}
\end{frame}
\begin{frame}
\frametitle{SCCs finden mittels DFS}
\framesubtitle{Idee des Algorithmus}
	\begin{itemize}
		\item Führe eine DFS im Graph durch.
		\item Besuchte Knoten erhalten zwei Nummern:
		\pause
			\begin{enumerate}
				\item \textbf{dfs\_num(u):} Speichert Schritt, in dem Knoten $u$ von DFS besucht wurde.
				\item \textbf{dfs\_low(u):}  $\min \left \{\textbf{dfs\_num(v)} | v \ erreichbar \ von \ u \right \}$
			\end{enumerate}
		\pause
		\item SCCs werden mittels dem von der DFS erzeugten Spannbaum gefunden
		\pause
		\item Wenn Backedge von einem Knoten $u$ zur Wurzel $r$ eines Teilbaums, sind alle Knoten auf \newline
		dem Weg zwischen $u$ und $v$ in einer SCC.
		%kurzes tafelbild dazu
	
	\end{itemize}
\end{frame}
\begin{frame}
	\frametitle{SCCs finden mittels DFS}
	\framesubtitle{DFS-Spannbaum}
	\includegraphics[scale = 0.6]<1-1>{spannbaum.pdf}
\end{frame}
\begin{frame}
	\frametitle{SCCs finden mittels DFS}
	\framesubtitle{Idee des Algorithmus}
	\begin{itemize}
		\item Führe eine DFS im Graph durch.
		\item Besuchte Knoten erhalten zwei Nummern:
		
		\begin{enumerate}
			\item \textbf{dfs\_num(u):} Speichert Schritt, in dem Knoten $u$ von DFS besucht wurde.
			\item \textbf{dfs\_low(u):}  $\min \left \{\textbf{dfs\_num(v)} | v \ erreichbar \ von \ u \right \}$
		\end{enumerate}
		
		\item SCCs werden mittels dem von der DFS erzeugten Spannbaum gefunden
		
		\item Wenn Backedge von einem Knoten $u$ zur Wurzel $r$ eines Teilbaums, sind alle Knoten auf \newline
		dem Weg zwischen $u$ und $v$ in einer SCC.
		%kurzes tafelbild dazu
		\pause
		\item Wenn \textbf{dfs\_low(v)} =  \textbf{dfs\_num(v)}, dann ist $v$ die "Wurzel" einer SCC
		\pause
		\item Knoten werden - sobald besucht - auf \textbf{STACK} gespeichert und wenn Wurzel gefunden, ausgegeben
	\end{itemize}
\end{frame}
\begin{frame}
	
	
	\frametitle{SCCs finden mittels DFS}
	\includegraphics[scale = 0.6, page=1]<1-1>{SCC_Beispiel_all.pdf}
	\includegraphics[scale = 0.6, page=2]<2-2>{SCC_Beispiel_all.pdf}
	\includegraphics[scale = 0.6, page=3]<3-3>{SCC_Beispiel_all.pdf}
	\includegraphics[scale = 0.6, page=4]<4-4>{SCC_Beispiel_all.pdf}
	\includegraphics[scale = 0.6, page=5]<5-5>{SCC_Beispiel_all.pdf}
	\includegraphics[scale = 0.6, page=6]<6-6>{SCC_Beispiel_all.pdf}
	\includegraphics[scale = 0.6, page=7]<7-7>{SCC_Beispiel_all.pdf}
	\includegraphics[scale = 0.6, page=8]<8-8>{SCC_Beispiel_all.pdf}
	\includegraphics[scale = 0.6, page=9]<9-9>{SCC_Beispiel_all.pdf}
	\includegraphics[scale = 0.6, page=10]<10-10>{SCC_Beispiel_all.pdf}
	\includegraphics[scale = 0.6, page=11]<11-11>{SCC_Beispiel_all.pdf}
	\includegraphics[scale = 0.6, page=12]<12-12>{SCC_Beispiel_all.pdf}
	\includegraphics[scale = 0.6, page=13]<13-13>{SCC_Beispiel_all.pdf}
	\includegraphics[scale = 0.6, page=14]<14-14>{SCC_Beispiel_all.pdf}
	\includegraphics[scale = 0.6, page=15]<15-15>{SCC_Beispiel_all.pdf}
	\includegraphics[scale = 0.6, page=16]<16-16>{SCC_Beispiel_all.pdf}
	
\end{frame}

\begin{frame}
	\frametitle{SCCs finden mittels DFS}
	\framesubtitle{Sourcecode}
	\lstset{basicstyle=\scriptsize, commentstyle=\color{mygreen},frame=single}
	\lstinputlisting[language=C++,firstline=17, lastline=37]{sourcecode/SCC_all.cpp}
\end{frame}

\begin{frame}
	\begin{itemize} 
		\item \textbf{findSCC(int u)} findet alle SCCs, die von Konten \textbf{u} aus erreichbar sind.
			\item Für vollständige Liste an SCCs \textbf{findSCC(int u)} für alle Knoten eines Graphen laufen lassen.
	\end{itemize}
\end{frame}
