\begin{frame}
	\Huge Bipartite Graphen
\end{frame}

\begin{frame}
	\frametitle{Bipartite Graphen}
	\framesubtitle{Definition}
	\begin{KITinfoblock}{Separatoren und Brücken in ungerichteten Graphen}
		Sei $G = (V,E)$ ein ungerichteter Graph.
	\begin{itemize}
		\item  $G$ heißt \textbf{Bipartite} falls sich seine Knoten in zwei disjunkte Teilmengen $A, B$ aufteilen lassen, so dass es zwischen den Knoten innerhalb einer Teilmenge keine Kanten gibt.
		\item  Das heißt, für jeder Kante $\{u,v\} \in E$ gilt entweder $u \in A$ und $v \in B$ oder $u \in B$ und $v \in A$.
	\end{itemize}
	\end{KITinfoblock}
\end{frame}
\begin{frame}
	\frametitle{Bipartite Graphen}
	\framesubtitle{Beispiel}
	\includegraphics[scale = 0.43]<1-1>{images/Bipartite/AB1}  
	\includegraphics[scale = 0.43]<2-2>{images/Bipartite/AB2}
\end{frame}

\begin{frame}
	\frametitle{Bipartite Graph Check}
	\framesubtitle{DFS Algorithmus}
	\includegraphics[scale = 0.7]<1-1>{images/Bipartite/t0}
	\includegraphics[scale = 0.7]<2-2>{images/Bipartite/t1}
	\includegraphics[scale = 0.7]<3-3>{images/Bipartite/t2}
	\includegraphics[scale = 0.7]<4-4>{images/Bipartite/t3}
	\includegraphics[scale = 0.7]<5-5>{images/Bipartite/t4}
	\includegraphics[scale = 0.7]<6-6>{images/Bipartite/t5}
	\includegraphics[scale = 0.7]<7-7>{images/Bipartite/t6}
	\includegraphics[scale = 0.7]<8-8>{images/Bipartite/t7}
	\includegraphics[scale = 0.7]<9-9>{images/Bipartite/t8}
	\includegraphics[scale = 0.7]<10-10>{images/Bipartite/t9}
	\includegraphics[scale = 0.7]<11-11>{images/Bipartite/t91}
\end{frame}

\begin{frame}
	\frametitle{Bipartite Graph Check}
	\framesubtitle{DFS Algorithmus}
	\lstinputlisting[style=customc++]{sourcecode/bipartiteDFS.cpp}	
\end{frame}

\begin{frame}
	\frametitle{Bipartite Graph Check}
	\framesubtitle{BFS Algorithmus}
	\lstinputlisting[style=customc++]{sourcecode/bipartiteBFS.cpp}	
\end{frame}