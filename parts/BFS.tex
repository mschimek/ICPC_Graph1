\begin{frame}
    \Huge Breitensuche
    \Large breadth-first search, BFS
\end{frame}
\begin{frame}
    \frametitle{Breitensuche}
    \begin{KITinfoblock}{Breitensuche}
        Traversieren der Knoten der Breite/der Entfernung zum Startknoten nach.\\[0.7em]
        \textbf{Idee}: Besuche den Startknoten, dann dessen Nachbarn, dann deren Nachbarn, usw...\\[0.7em]
        \textbf{Implementierung}: Queue und besuchte Knoten markieren
        \begin{enumerate}
            \item Startknoten markieren und in die Queue einreihen
            \item den ersten Knoten \textbf{u} aus der Queue nehmen
            \item alle nicht markierten Nachbarn von \textbf{u} markieren und einreihen
            \item gehe zu \textit{2.} wenn Queue nicht leer sonst fertig
        \end{enumerate}
    \end{KITinfoblock}
\end{frame}

\begin{frame}
    \frametitle{Breitensuche}
    \framesubtitle{Beispiel}
    \begin{center}
        \includegraphics[scale=0.50]<1-1>{bfs00.png}
    \end{center}

    Startknoten s = 6 \hspace{0.5cm} Queue q = \{6\} \hspace{0.5cm} Markierte Knotenmege d = \{6\}
\end{frame}

\begin{frame}
    \frametitle{Breitensuche}
    \framesubtitle{Beispiel}
    \begin{center}
        \includegraphics[scale=0.50]<1-1>{bfs01.png}
    \end{center}

    u = 6 \hspace{1cm} q = \{\} \\ d = \{6\}
\end{frame}
\begin{frame}
    \frametitle{Breitensuche}
    \framesubtitle{Beispiel}
    \begin{center}
        \includegraphics[scale=0.50]<1-1>{bfs02.png}
    \end{center}

    u = 6 \hspace{1cm} q = \{2\} \\ d = \{6, 2\}
\end{frame}
\begin{frame}
    \frametitle{Breitensuche}
    \framesubtitle{Beispiel}
    \begin{center}
        \includegraphics[scale=0.50]<1-1>{bfs03.png}
    \end{center}

    u = 6 \hspace{1cm} q = \{2, 7\} \\ d = \{6, 2, 7\}
\end{frame}
\begin{frame}
    \frametitle{Breitensuche}
    \framesubtitle{Beispiel}
    \begin{center}
        \includegraphics[scale=0.50]<1-1>{bfs04.png}
    \end{center}

    u = 6 \hspace{1cm} q = \{2, 7, 9\} \\ d = \{6, 2, 7, 9\}
\end{frame}
\begin{frame}
    \frametitle{Breitensuche}
    \framesubtitle{Beispiel}
    \begin{center}
        \includegraphics[scale=0.50]<1-1>{bfs05.png}
    \end{center}

    u = 2 \hspace{1cm} q = \{7, 9\} \\ d = \{6, 2, 7, 9\}
\end{frame}
\begin{frame}
    \frametitle{Breitensuche}
    \framesubtitle{Beispiel}
    \begin{center}
        \includegraphics[scale=0.50]<1-1>{bfs06.png}
    \end{center}

    u = 2 \hspace{1cm} q = \{7, 9, 1, 3\} \\ d = \{6, 2, 7, 9, 1, 3\}
\end{frame}
\begin{frame}
    \frametitle{Breitensuche}
    \framesubtitle{Beispiel}
    \begin{center}
        \includegraphics[scale=0.50]<1-1>{bfs07.png}
    \end{center}

    u = 7 \hspace{1cm} q = \{9, 1, 3\} \\ d = \{6, 2, 7, 9, 1, 3\}
\end{frame}
\begin{frame}
    \frametitle{Breitensuche}
    \framesubtitle{Beispiel}
    \begin{center}
        \includegraphics[scale=0.50]<1-1>{bfs08.png}
    \end{center}

    u = 9 \hspace{1cm} q = \{1, 3, 8, 10, 11\} \\ d = \{6, 2, 7, 9, 1, 3, 8, 10, 11\}
\end{frame}
\begin{frame}
    \frametitle{Breitensuche}
    \framesubtitle{Beispiel}
    \begin{center}
        \includegraphics[scale=0.50]<1-1>{bfs09.png}
    \end{center}

    u = 11 \hspace{1cm} q = \{0, 5, 4\} \\ d = \{6, 2, 7, 9, 1, 3, 8, 10, 11, 0, 5, 4\}
\end{frame}
\begin{frame}
    \frametitle{Breitensuche}
    \framesubtitle{Beispiel}
    \begin{center}
        \includegraphics[scale=0.50]<1-1>{bfs10.png}
    \end{center}

    q = \{\} \\ d = \{6, 2, 7, 9, 1, 3, 8, 10, 11, 0, 5, 4\}
\end{frame}

\begin{frame}
    \frametitle{Breitensuche}
    \framesubtitle{weiteres}
    \begin{itemize}
        \item \textbf{Laufzeit}: \( \mathcal{O}(|V| + |E|)\) bzw. \( \mathcal{O}(|V|^2)\)
        \item \textbf{Speicher}: \( \mathcal{O}(|V|)\) da alle endeckten Knoten gespeichert werden
        \item statt zu markieren, speichere das "Level" (Level Elternknoten + 1) von \textbf{u} und erhälte die Entfernung zu \textbf{s} \\\(\iff\) Länge des kürzesten Pfades von \textbf{s} nach \textbf{u}
    \end{itemize}
\end{frame}
\begin{frame}
    \frametitle{Breitensuche}
    \framesubtitle{Code}
    \lstinputlisting[style=customc++]{sourcecode/bfs.cpp}
\end{frame}
