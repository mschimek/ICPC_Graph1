\begin{frame}
		\Huge Iterative Tiefensuche
\end{frame}
	
\begin{frame}
	\frametitle{Iterative Tiefensuche}
	\begin{KITinfoblock}{Begrenzt Tiefensuche}
	\begin{itemize}
		\item \textbf{Begrenzt Tiefensuche}: wird als eine DFS Suche gebaut, die nicht tiefer als eine Tiefe $T$ sucht (tiefere Knoten werden ignoriert).
		\item Für sich allein nur in besonderen Situationen nützlich, ist aber eine wichtige Komponente in Iterative Tiefensuche.
	\end{itemize}
	\end{KITinfoblock}
\end{frame}

\begin{frame}
	\frametitle{Iterative Tiefensuche}
	\framesubtitle{Rekursive begrenzt DFS Algorithmus}
	\lstset{basicstyle=\scriptsize, commentstyle=\color{mygreen},frame=single}
	\lstinputlisting[language=C++,lastline=17]{sourcecode/iterativeDFS.cpp}	
\end{frame}

\begin{frame}
	\frametitle{Iterative Tiefensuche}
	\begin{KITinfoblock}{Iterative Tiefensuche}
		\begin{itemize}
			\item Idee: Führe eine Folge begrentzter Tiefensuchen mit ansteigender Tiefengrenze, bis eine Lösung gefunden wird.
			
			\item Sei $b$ den durchschnittliche Grad, und $d$ die minimale Lösungstiefe eines gegebenen Graph $G$.
			  
			\vspace{0.08in}
			Der Zeitaufwand der iterativen Tiefensuche ist dann
			\linebreak[2]
			$(d+1) + db + (d-1)b^2 + (d-2)b^3 + ... + 2b^{d-1} + b^d = \mathcal{O}(b^d)$
			
			und der Speicheraufwand beträgt $\mathcal{O}(bd)$.
			\begin{itemize}
				\item BFS $\Rightarrow 1 + b + b^2 + ... + b^d = \mathcal{O}(b^d)$
			\end{itemize}			
		\end{itemize}
	\end{KITinfoblock}
	\pause
	\lstset{basicstyle=\scriptsize, commentstyle=\color{mygreen},frame=single}
	\lstinputlisting[language=C++,firstline=19]{sourcecode/iterativeDFS.cpp}
	
\end{frame}

\begin{frame}
	\frametitle{Iterative Tiefensuche}
	\includegraphics[scale = 0.6]<1-1>{images/Iterative/d0.pdf}
	\includegraphics[scale = 0.6]<2-2>{images/Iterative/d1.pdf}
	\includegraphics[scale = 0.6]<3-3>{images/Iterative/d2.pdf}
	\includegraphics[scale = 0.6]<4-4>{images/Iterative/d3.pdf}
	\includegraphics[scale = 0.6]<5-5>{images/Iterative/d4.pdf}
	\includegraphics[scale = 0.6]<6-6>{images/Iterative/d5.pdf}
	\includegraphics[scale = 0.6]<7-7>{images/Iterative/d6.pdf}
	\includegraphics[scale = 0.6]<8-8>{images/Iterative/d7.pdf}
	\includegraphics[scale = 0.6]<9-9>{images/Iterative/d8.pdf}
	\includegraphics[scale = 0.6]<10-10>{images/Iterative/d9.pdf}
	\includegraphics[scale = 0.6]<11-11>{images/Iterative/d10.pdf}
	\includegraphics[scale = 0.6]<12-12>{images/Iterative/d11.pdf}
	\includegraphics[scale = 0.6]<13-13>{images/Iterative/d12.pdf}
	\includegraphics[scale = 0.6]<14-14>{images/Iterative/d13.pdf}
	\includegraphics[scale = 0.6]<15-15>{images/Iterative/d14.pdf}
\end{frame}

\begin{frame}
	\frametitle{Iterative Tiefensuche}
	\Large Warte mal: \normalsize jede Iteration besucht die vorherige Knoten noch einmal!
	Warum dann nicht BFS oder DFS benutzen?
\end{frame}

\begin{frame}
	\begin{KITinfoblock}{Warum dann nicht BFS oder DFS}
	 \begin{itemize}
	 	\item Einige Graphen können mit herkömmlichen Suchverfahren nicht durchgelaufen werden, bspw. implizite Graphen, die unendlich Groß sind, und können im Speicher nicht gehalten werden. \vspace{0.2in}

		\includegraphics[scale = 0.49]<1-1>{images/Iterative/grid2.pdf}
		
	\end{itemize}
	\end{KITinfoblock}
\end{frame}

\begin{frame}
	\frametitle{Iterative Tiefensuche}
	\begin{KITexampleblock}{Schach}
		\begin{itemize}
			\item Auch sinnvoll für Probleme, die einen sehr hohen branching factor $b$ haben, z.B.: \textbf{Schach}
			\item Im Durchschnitt hat ein Schachspieler für jeder Position 35\textasciitilde38 mögliche Züge. Das heißt, nach 5 Züge werden \textasciitilde$10^7$ Positionen möglich, nach 10 Züge \textasciitilde$10^{15}$.
			\item Adaptierte iterative Tiefensuche berechnet den besten Zug bis entweder der Zeit abgelaufen ist oder die maximale Suchtiefe erreicht ist.
			\item Suche kann mit Heuristiken verbessert werden, so dass die besten Richtungen ersten erforscht sind.
		\end{itemize}
	\end{KITexampleblock}
\end{frame}

\begin{frame}
	\frametitle{Iterative Tiefensuche}
	\includegraphics[scale = 0.4]<1-1>{images/Iterative/chess1}
	\includegraphics[scale = 0.4]<2-2>{images/Iterative/chess2}
	\includegraphics[scale = 0.4]<3-3>{images/Iterative/chess3}
\end{frame}

\begin{frame}
	\frametitle{Iterative Tiefensuche}
	\Large Source code: \href{www.craftychess.com/crafty-23.4.zip}{www.craftychess.com/crafty-23.4.zip}
\end{frame}